\section{Introducci\'on}
La repulsi\'on de polos magn\'eticos similares generan una fuerza considerable. El control de dicha fuerza para generar movimiento es
una t\'ecnica empleada bastante en la industria de transportaci\'on, un ejemplo de esto son los trenes bala que operan mediante series
 de imanes ubicados en los rieles.

\section{Objetivos}
\begin{itemize}
 \item Construir un motor de pulsos
 \item Dise\~nar y construir un tac\'ometro digital
\end{itemize}

\section{Descripci\'on y Presentaci\'on}
El dise\~no del motor busc\'o reducir la fricci\'on del rotor a lo m\'inimo, por ello se opt\'o por reducir la superficie de contacto
entre el rotor y los soportes. Para ello se emple\'o una aguja como rotor, la cual concentra todo el peso del sistema en un
espacio m\'inimo. Se emple\'o un electroim\'an de tama\~no considerable para generar el campo magn\'etico, y se colocaron
cuatro imanes de neodimio en el rotor para asegurar la conservaci\'on de momento del sistema.\\

El dise\~no del tac\'ometro comprendi\'o varias etapas: dise\~no del circuito, programaci\'on, simulaci\'on y pruebas f\'isicas.
Inicialmente se desarroll\'o c\'odigo en C para controlar mediante un PIC16F887 un  display de cristal l\'iquido. Sin embargo,
el programador disponible en el campus dej\'o de operar por fallas de la fuente de alimentaci\'on.