\section{Comentarios, observaciones y conclusiones}
\subsection{Isaac Ayala Lozano}
\subsubsection{Comentarios}
La parte m\'as dif\'icil del circuito fue el inductor,
puesto que las bobinas elaboradas por m\'i no produc\'ian
la inductancia necesaria para que el sistema funcione.
Fue necesario adquirir una bobina comercial que se encontrara
dentro de las especificaciones del circuito.
\subsubsection{Observaciones}
La distancia de detecci\'on del modelo resulta muy peque\~na
por la forma que posee el inductor. Esto podr\'ia compensarse
al reemplazar dicho inductor con otro del mismo valor de 
inductancia, pero con una gemoetr\'ia diferente.
\subsubsection{Conclusiones}
Los cambios en el campo y flujo magn\'etico de un sistema 
pueden verse reflejados en los cambios de corriente y voltaje 
del sistema.
