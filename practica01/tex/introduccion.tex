\section{Introducci\'on}
La detecci\'on de metales ha sido una actividad que ha fascinado a una parte de la 
sociedad, pues \'esta implica en  algunos casos el descubrimiento de objetos de 
valor. Esta pr\'actica plantea el dise\~no y fabricaci\'on de un detector de metales
capaz de diferenciar entre los materiales que detecte.
\section{Objetivos}
\begin{itemize}
 \item Comprender el funcionamiento de un detector de metales
 \item Dise\~nar el circuito electr\'onico de un detector de metales
 \item Fabricar un modelo funcional a escala natural
\end{itemize}

\section{Descripci\'on y Presentaci\'on}
El proyecto se dividi\'o en tres partes: la investigaci\'on, el dise\~no 
y la manufactura. Para la parte de investigaci\'on se encontraron estudios
realizados por miembros de la IEEE, los cu\'ales describ\'ian t\'ecnicas de estudio
para los sensores del tipo BFO y la obtenci\'on de ecuaciones para describir su
comportamiento. Tambi\'en se encontraron an\~n\'alisis del funcionamiento de 
los sensores empleados en la industria alimenticia.

El dise\~no emple\'o el modelo de variaci\'on de frecuencia que utiliza un LM555 
como base. El proceso de manufactura fue el termoformado del material para 
darle la forma deseada.
