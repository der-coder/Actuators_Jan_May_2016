\section{Introducci\'on}
En esta pr\'actica se pretende demostrar la posibilidad de convertir cambios en el campo magnti\'eco de un objeto en se\~nales el\'ectricas y 
\'estas a su vez en ondas sonoras.
\section{Objetivos}
\begin{itemize}
 \item Construir un instrumento musical que opere con se\~nales el\'etricas
 \item Crear un amplificador de audio
 \item Elaborar una pastilla de guitarra el\'ectrica
\end{itemize}

\section{Descripci\'on y Presentaci\'on}
La practica tuvo cuatro etapas:
\begin{enumerate}
 \item Construcci\'on del amplificador
 \item Embobinado
 \item Manufactura del instrumento
 \item Prueba del instrumento
\end{enumerate}

La etapa del amplificador consisti\'o en la adquisici\'on de materiales y la elaboraci\'on del circuito de referencia provisto en la hoja de 
especificaciones del amplificador operacional (LM386N-1). Para operar requiere de una fuente de alimentaci\'on de 9 V, y su funcionamiento se 
comprob\'o con dos guitarras el\'ectricas comerciales.\\

El embobinado requiri\'o de la elaboraci\'on de una base con varios polos met\'alicos para aumentar la magnitud del flujo magn\'etico resultante. 
Se emplearon tornillos comerciales y alambre calibre 33. El proceso de embobinado tom\'o alrededor de ocho horas.\\

Para la creaci\'on del instrumento se emple\'o un mueble previamente adquirido con la forma deseada, pues \'esto permiti\'o un ahorro de dos horas 
en la manufactura total. Se removieron los elementos sobrantes como ornamentos y ganchos, dejando solamente la madera en piezas separadas. Dos de 
ellas (la cabeza y el m\'astil) requirieron de un desbaste de material con el router para que el producto final mantuviera un tama\~no similar a 
aquel de las guitarras comerciales. \\

Se removieron la capa de barniz y pintura originales, aplicando un proceso de lijado por dos horas para preparar las piezas para darles el nuevo 
acabado. La caja y la cabeza fueron pintadas de color negro empleando tres capas de tinta en alcohol y despu\'es una capa de barniz transparente. 
El m\'astil recibi\'o dos capas de barniz transparente. Para complementar el trabajo actual, se a\~nadieron clavijas, una cejilla, una cejuela y 
una salida de 6.3 mm para asegurar que el funcionamiento del instrumento fuera lo m\'as cercano al modelo de referencia.