\section{An\'alisis de resultados}
La guitarra el\'ectrica funciona a base de cambios en el campo magn\'etico que la pastilla genera. Este campo magn\'etico se produce en respuesta a 
la interacci\'on de la bobina elaborada y los imanes adquiridos. Las cuerdas, al ser de metal, producen perturbaciones en el campo magn\'etico al 
comenzar a vibrar. Estos cambios ocasionan que una corriente el\'ectrica se produzca a trav\'es del embobinado, qu se transmite al amplificador.
El amplificador se encarga de procesar la se\~nal y aumentarla antes de enviarla a la bocina, resultando en un sonido audible para el ser humano.