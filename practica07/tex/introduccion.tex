\section{Introducci\'on}
Los motores de corriente directa tienen usos limitados, pues el torque que pueden proveer est\'a limitado severamente; por
ello, la categor\'ia de motores de corriente alterna han dominado el mercado. Los motores de corriente alterna est\'an 
dise\~nados para trabajar con cargas mucho m\'as grandes, operar con voltajes y corrientes superiores a las limitaciones 
de los motores DC, y para reducir el costo de mantenimiento que requieren.

\section{Objetivos}
\begin{itemize}
 \item Dise\~nar y construir un motor as\'incrono
 \item Adaptar el motor as\'incrono para funcionar como motor s\'incrono
\end{itemize}

\section{Descripci\'on y Presentaci\'on}
La construcci\'on del modelo funcional requiri\'o investigar las diferentes configuraciones de motores monof\'asicos, esto 
con el objetivo de comprender el funcionamiento de dicho circuito, y seleccionar la configuraci\'on m\'as adecuada.
Se opt\'o por utilizar dos capacitores de $4700 \mu F$ en antiserie, conectados a una bobina auxiliar. Se emplearon dos 
bobinas con igual cantidad de vueltas. \'Estas se conectaron directamente a un trasformador de $24 V_{rms}$ y $5 A$.