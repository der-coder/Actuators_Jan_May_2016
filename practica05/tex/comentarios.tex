\section{Comentarios, observaciones y conclusiones}
\subsection{Isaac Ayala Lozano}
\subsubsection{Comentarios}
El desarrollo de la pr\'actica implic\'o invertir m\'as tiempo del esperado, pues fue necesario hacer tres veces las bobinas.
Esto se debi\'o a que el total de vueltas que se emple\'o en las primeras dos ocasiones fue muy bajo.

\subsubsection{Observaciones}
El n\'umero de vueltas de cada bobina determin\'o si la configuraci\'on del motor funcionar\'ia o no. Tambi\'en se not\'o 
la formaci\'on de \'oxido en el eje del rotor debido al arco el\'ctrico que se form\'o.

\subsubsection{Conclusiones}
Es necesario tomar en cuenta el n\'umero de vueltas y las restricciones f\'isicas del modelo para poder dise\~nar motores 
de corriente directa.
