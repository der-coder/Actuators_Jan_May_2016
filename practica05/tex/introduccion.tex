\section{Introducci\'on}
El motor de corriente directa (DC o CC) es una m\'aquina el\'ectrica, dise\~nada para convertir la energ\'ia el\'ectrica en trabajo
mec\'anico. Su funcionamiento depende del fen\'omeno de inducci\'on magn\'etica en materiales conductores.
A trav\'es de la repulsi\'on del elemento conductor con polos magn\'eticos fijos, \'este comienza a girar; generando
as\'i el movimiento del sistema.

\section{Objetivos}
\begin{itemize}
 \item Comprender el funcionamiento de las diferentes configuraciones del motor DC
 \item Presentar modelos funcionales de las diferentes configuraciones
\end{itemize}

\section{Descripci\'on y Presentaci\'on}
Se investigaron las diferentes maneras de contruir el motor de corriente directa. Se encontr\'o que 
existen cuatro configuraciones principales: excitaci\'on separada, autoexitaci\'on, en serie, y 
compuesto. Cada una de ellas requiere un esquema de conexiones distintas. Se dise\~n\'o tambi\'en un motor 
adicional para desplazar un peso de 500 gramos, pero la implementaci\'on no fue excitosa.